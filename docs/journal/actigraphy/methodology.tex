\section{Methodology}

\subsection{Data Collection Framework}

\subsubsection{Consumer-Grade Wearable Device Selection}

Data collection utilized consumer-grade wearable devices providing continuous physiological monitoring capabilities. Primary devices included Oura Ring (Generation 3), Garmin Vivosmart series, and Fitbit Charge series, selected for their photoplethysmography (PPG) and tri-axial accelerometry sensor integration.

\begin{definition}[Device Specification Requirements]
Selected devices met the following technical specifications:
\begin{align}
f_{\text{PPG}} &\geq 25 \text{ Hz} \quad \text{(PPG sampling frequency)} \\
f_{\text{accel}} &\geq 50 \text{ Hz} \quad \text{(accelerometry sampling frequency)} \\
\lambda_{\text{PPG}} &= 525-590 \text{ nm} \quad \text{(green light wavelength)} \\
\text{Resolution}_{\text{accel}} &\leq 0.01 \text{ g} \quad \text{(acceleration resolution)}
\end{align}
\end{definition}

\subsubsection{Data Acquisition Protocol}

Continuous monitoring was conducted over periods ranging from 30 to 180 days per subject. Data acquisition followed standardized protocols:

\begin{enumerate}
\item Device calibration using manufacturer-specified procedures
\item Baseline measurement period of 7 days for individual parameter estimation
\item Continuous 24-hour monitoring with device removal only for charging
\item Data synchronization at 24-hour intervals via manufacturer APIs
\item Quality control assessment using signal-to-noise ratio thresholds
\end{enumerate}

\begin{definition}[Data Quality Criteria]
Measurements were included in analysis if they satisfied:
\begin{align}
\text{SNR}_{\text{PPG}} &> 10 \text{ dB} \\
\text{Coverage} &> 95\% \text{ per 24-hour period} \\
\text{Artifact Rate} &< 5\% \text{ per measurement window}
\end{align}
\end{definition}

\subsection{Multi-Scale Oscillatory Analysis Implementation}

\subsubsection{Oscillatory Component Extraction}

Raw sensor signals were decomposed into oscillatory components using empirical mode decomposition (EMD) and wavelet transform methods:

\begin{definition}[Empirical Mode Decomposition Implementation]
For a signal $x(t)$, EMD produces intrinsic mode functions (IMFs):
\begin{equation}
x(t) = \sum_{i=1}^{N} \text{IMF}_i(t) + r_N(t)
\end{equation}
where $\text{IMF}_i(t)$ represents the $i$-th intrinsic mode function and $r_N(t)$ denotes the final residue.
\end{definition}

\begin{definition}[Continuous Wavelet Transform Application]
Wavelet coefficients were computed as:
\begin{equation}
W(a,b) = \frac{1}{\sqrt{a}} \int_{-\infty}^{\infty} x(t) \psi^*\left(\frac{t-b}{a}\right) dt
\end{equation}
where $\psi(t)$ represents the mother wavelet (Morlet wavelet), $a$ denotes scale parameter, $b$ indicates translation parameter, and $*$ represents complex conjugation.
\end{definition}

\subsubsection{Multi-Scale Frequency Analysis}

The 11-scale hierarchical architecture was analyzed through frequency domain decomposition:

\begin{algorithm}
\caption{Multi-Scale Frequency Decomposition}
\begin{algorithmic}
\Procedure{MultiScaleAnalysis}{RawSignal, FrequencyBands}
    \For{$i = 0$ to $10$}
        \State $f_{\text{band}} \leftarrow$ FrequencyBands[$i$]
        \State $\text{FilteredSignal}_i \leftarrow$ BandpassFilter(RawSignal, $f_{\text{band}}$)
        \State $A_i(t) \leftarrow$ HilbertTransform($\text{FilteredSignal}_i$).amplitude
        \State $\phi_i(t) \leftarrow$ HilbertTransform($\text{FilteredSignal}_i$).phase
        \State $\mathbf{\Psi}_i \leftarrow [A_i(t), \phi_i(t)]$
    \EndFor
    \State \Return $\{\mathbf{\Psi}_i\}_{i=0}^{10}$
\EndProcedure
\end{algorithmic}
\end{algorithm}

\subsubsection{Coupling Strength Quantification}

Inter-scale coupling was quantified using phase-amplitude coupling (PAC) and phase-phase coupling (PPC) measures:

\begin{definition}[Phase-Amplitude Coupling Implementation]
PAC between scales $i$ and $j$ was computed as:
\begin{equation}
\text{PAC}_{ij} = \left| \frac{1}{N} \sum_{n=1}^{N} A_j[n] e^{i\phi_i[n]} \right|
\end{equation}
where $A_j[n]$ represents amplitude of scale $j$ at time $n$ and $\phi_i[n]$ denotes phase of scale $i$ at time $n$.
\end{definition}

\begin{definition}[Phase-Phase Coupling Implementation]
PPC between scales $i$ and $j$ was calculated as:
\begin{equation}
\text{PPC}_{ij} = \left| \frac{1}{N} \sum_{n=1}^{N} e^{i(\phi_i[n] - \phi_j[n])} \right|
\end{equation}
\end{definition}

\subsection{Activity-Sleep Mirror Analysis}

\subsubsection{Error Accumulation Calculation}

Metabolic error accumulation was computed from MET time series data using numerical integration:

\begin{algorithm}
\caption{Error Accumulation Computation}
\begin{algorithmic}
\Procedure{ComputeErrorAccumulation}{METTimeSeries, $\alpha$, $\text{MET}_{\text{baseline}}$}
    \State $E_{\text{total}} \leftarrow 0$
    \State $\Delta t \leftarrow$ SamplingInterval(METTimeSeries)
    \For{each $\text{MET}(t)$ in METTimeSeries}
        \State $\text{ExcessMET} \leftarrow \max(0, \text{MET}(t) - \text{MET}_{\text{baseline}})$
        \State $\Delta E \leftarrow \alpha \times \text{ExcessMET} \times \Delta t$
        \State $E_{\text{total}} \leftarrow E_{\text{total}} + \Delta E$
    \EndFor
    \State \Return $E_{\text{total}}$
\EndProcedure
\end{algorithmic}
\end{algorithm}

\subsubsection{Sleep Cleanup Capacity Assessment}

Sleep cleanup capacity was calculated from hypnogram data and sleep efficiency metrics:

\begin{definition}[Hypnogram Processing]
Sleep stage durations were extracted from 5-minute resolution hypnogram strings:
\begin{equation}
T_{\text{stage}} = \frac{\text{Count}(\text{stage})}{12} \text{ hours}
\end{equation}
where Count(stage) represents the number of 5-minute epochs in the specified sleep stage.
\end{definition}

\begin{algorithm}
\caption{Cleanup Capacity Calculation}
\begin{algorithmic}
\Procedure{ComputeCleanupCapacity}{Hypnogram, SleepEfficiency, $\beta_{\text{deep}}$, $\beta_{\text{REM}}$}
    \State $T_{\text{deep}} \leftarrow$ ExtractStageDuration(Hypnogram, 'D')
    \State $T_{\text{REM}} \leftarrow$ ExtractStageDuration(Hypnogram, 'R')
    \State $\eta_{\text{sleep}} \leftarrow$ SleepEfficiency / 100
    \State $C_{\text{deep}} \leftarrow \beta_{\text{deep}} \times T_{\text{deep}} \times \eta_{\text{sleep}}$
    \State $C_{\text{REM}} \leftarrow \beta_{\text{REM}} \times T_{\text{REM}} \times \eta_{\text{sleep}}$
    \State $C_{\text{total}} \leftarrow C_{\text{deep}} + C_{\text{REM}}$
    \State \Return $C_{\text{total}}$
\EndProcedure
\end{algorithmic}
\end{algorithm}

\subsubsection{Mirror Coupling Validation}

Mirror coupling coefficients were computed and validated through statistical analysis:

\begin{definition}[Mirror Coefficient Calculation]
For each activity-sleep pair, the mirror coefficient was computed as:
\begin{equation}
\text{MirrorCoeff} = \frac{C_{\text{total}}}{E_{\text{total}}}
\end{equation}
with perfect coupling indicated by MirrorCoeff ≈ 1.0.
\end{definition}

\subsection{S-Entropy Coordinate Transformation}

\subsubsection{4D Coordinate Computation}

S-entropy coordinates were computed from processed sensor data using the following implementation:

\begin{algorithm}
\caption{S-Entropy Coordinate Transformation}
\begin{algorithmic}
\Procedure{ComputeSEntropyCoords}{ActivityData, HRVData, ContextData}
    \State // Knowledge Coordinate
    \State $p_i \leftarrow$ ComputeActivityStateProbabilities(ActivityData)
    \State $S_{\text{knowledge}} \leftarrow -\sum_i p_i \log_2(p_i) + \alpha \times \text{SensorInfoContent}$
    
    \State // Time Coordinate  
    \State $\phi_{\text{activity}}(t) \leftarrow$ ExtractInstantaneousPhase(ActivityData)
    \State $S_{\text{time}} \leftarrow \frac{1}{T} \int_0^T |\frac{d\phi_{\text{activity}}}{dt}| dt$
    
    \State // Entropy Coordinate
    \State $\rho_k \leftarrow$ ComputeOscillatoryTerminationDensity(ActivityData)
    \State $S_{\text{entropy}} \leftarrow -\sum_k \rho_k \log_2(\rho_k)$
    
    \State // Context Coordinate
    \State $S_{\text{context}} \leftarrow \beta_1 \times \text{HRV} + \beta_2 \times \text{Temp} + \beta_3 \times \text{CircPhase}$
    
    \State \Return $[S_{\text{knowledge}}, S_{\text{time}}, S_{\text{entropy}}, S_{\text{context}}]$
\EndProcedure
\end{algorithmic}
\end{algorithm}

\subsubsection{Coordinate Space Navigation}

Navigation through S-entropy coordinate space was implemented using gradient-based optimization:

\begin{definition}[Navigation Velocity Computation]
The velocity field in S-entropy space was computed as:
\begin{equation}
\vec{v}_S = -\nabla_S \mathcal{E}(\vec{S}) + \mathbf{F}_{\text{external}}
\end{equation}
where $\mathcal{E}(\vec{S})$ represents the energy function and $\mathbf{F}_{\text{external}}$ denotes external forcing terms.
\end{definition}

\subsection{Comprehensive Activity Analysis Framework}

\subsubsection{Seven-Domain Analysis Implementation}

The comprehensive activity analysis was implemented through seven specialized modules:

\begin{enumerate}
\item \textbf{Basic Metrics Module}: Step count, distance, active minutes, sedentary time analysis
\item \textbf{Energy Expenditure Module}: Total daily energy expenditure, basal metabolic rate calculation
\item \textbf{Intensity-Based Module}: Activity intensity classification and pattern analysis
\item \textbf{Movement Pattern Module}: Locomotor pattern recognition and gait analysis
\item \textbf{Postural Analysis Module}: Body position and orientation pattern analysis
\item \textbf{Heart Rate Variability Module}: Time-domain and frequency-domain HRV analysis
\item \textbf{Coupling Analysis Module}: Cross-domain coupling strength quantification
\end{enumerate}

\subsubsection{Statistical Analysis Framework}

Statistical validation employed multiple analytical approaches:

\begin{definition}[Correlation Analysis]
Pearson correlation coefficients were computed as:
\begin{equation}
r_{xy} = \frac{\sum_{i=1}^{n}(x_i - \bar{x})(y_i - \bar{y})}{\sqrt{\sum_{i=1}^{n}(x_i - \bar{x})^2}\sqrt{\sum_{i=1}^{n}(y_i - \bar{y})^2}}
\end{equation}
where $x_i$ and $y_i$ represent paired observations and $\bar{x}$, $\bar{y}$ denote sample means.
\end{definition}

\begin{definition}[Cross-Validation Protocol]
K-fold cross-validation was implemented with $K = 10$:
\begin{equation}
\text{CV Error} = \frac{1}{K} \sum_{k=1}^{K} \mathcal{L}(\mathcal{M}_k, \mathcal{D}_k^{\text{test}})
\end{equation}
where $\mathcal{M}_k$ represents the model trained on fold $k$ and $\mathcal{L}$ denotes the loss function.
\end{definition}

\subsection{Predictive Modeling Implementation}

\subsubsection{Multiple Regression Analysis}

Predictive models were developed using multiple regression with polynomial terms:

\begin{definition}[Polynomial Regression Model]
The general form of polynomial regression models was:
\begin{equation}
y = \beta_0 + \sum_{i=1}^{p} \beta_i x_i + \sum_{i=1}^{p} \sum_{j=i}^{p} \beta_{ij} x_i x_j + \epsilon
\end{equation}
where $\beta_i$ represent regression coefficients, $x_i$ denote predictor variables, and $\epsilon$ represents error terms.
\end{definition}

\subsubsection{Model Selection and Validation}

Model selection employed information criteria and cross-validation:

\begin{definition}[Akaike Information Criterion]
Model selection utilized AIC:
\begin{equation}
\text{AIC} = 2k - 2\ln(\mathcal{L})
\end{equation}
where $k$ represents the number of parameters and $\mathcal{L}$ denotes the likelihood function.
\end{definition}

\begin{definition}[Bayesian Information Criterion]
BIC was computed as:
\begin{equation}
\text{BIC} = k\ln(n) - 2\ln(\mathcal{L})
\end{equation}
where $n$ represents the sample size.
\end{definition}

\subsection{Oscillatory Pattern Recognition}

\subsubsection{Fourier Analysis Implementation}

Frequency domain analysis was performed using Fast Fourier Transform (FFT):

\begin{definition}[Power Spectral Density Computation]
Power spectral density was computed as:
\begin{equation}
S_{xx}(f) = \frac{1}{f_s N} \left| \sum_{n=0}^{N-1} x[n] e^{-j2\pi fn/f_s} \right|^2
\end{equation}
where $f_s$ represents sampling frequency, $N$ denotes signal length, and $x[n]$ represents discrete signal samples.
\end{definition}

\subsubsection{Phase Coherence Analysis}

Phase coherence between signals was quantified using the phase-locking value:

\begin{definition}[Phase-Locking Value]
PLV between signals $x$ and $y$ was computed as:
\begin{equation}
\text{PLV}_{xy} = \left| \frac{1}{N} \sum_{n=1}^{N} e^{i(\phi_x[n] - \phi_y[n])} \right|
\end{equation}
where $\phi_x[n]$ and $\phi_y[n]$ represent instantaneous phases extracted using the Hilbert transform.
\end{definition}

\subsection{Data Processing Pipeline}

\subsubsection{Quality Control and Preprocessing}

Raw sensor data underwent systematic quality control and preprocessing:

\begin{algorithm}
\caption{Data Preprocessing Pipeline}
\begin{algorithmic}
\Procedure{PreprocessData}{RawData}
    \State // Artifact Detection
    \State ArtifactMask $\leftarrow$ DetectArtifacts(RawData, ThresholdParams)
    \State CleanData $\leftarrow$ RemoveArtifacts(RawData, ArtifactMask)
    
    \State // Signal Filtering
    \State FilteredData $\leftarrow$ ButterworthFilter(CleanData, CutoffFreqs)
    
    \State // Normalization
    \State NormalizedData $\leftarrow$ ZScoreNormalization(FilteredData)
    
    \State // Interpolation for Missing Values
    \State CompleteData $\leftarrow$ CubicSplineInterpolation(NormalizedData)
    
    \State \Return CompleteData
\EndProcedure
\end{algorithmic}
\end{algorithm}

\subsubsection{Temporal Alignment and Synchronization}

Multi-sensor data streams were temporally aligned using cross-correlation:

\begin{definition}[Cross-Correlation Alignment]
Temporal offset between signals was determined by:
\begin{equation}
\tau_{\text{optimal}} = \arg\max_\tau \sum_{n} x[n] y[n+\tau]
\end{equation}
where $\tau_{\text{optimal}}$ represents the optimal time delay for signal alignment.
\end{definition}

\subsection{Computational Implementation}

\subsubsection{Software Framework}

Analysis was implemented using Python 3.8+ with specialized libraries:

\begin{itemize}
\item \textbf{NumPy 1.21+}: Numerical computations and array operations
\item \textbf{SciPy 1.7+}: Signal processing and statistical analysis
\item \textbf{Pandas 1.3+}: Data manipulation and time series analysis
\item \textbf{Scikit-learn 1.0+}: Machine learning and cross-validation
\item \textbf{Matplotlib 3.5+}: Visualization and plotting
\item \textbf{Seaborn 0.11+}: Statistical visualization
\end{itemize}

\subsubsection{Computational Complexity}

Algorithm complexity was optimized for real-time processing:

\begin{definition}[Computational Complexity Bounds]
Key algorithms achieved the following complexity bounds:
\begin{align}
\text{FFT Analysis} &: O(N \log N) \\
\text{Wavelet Transform} &: O(N \log N) \\
\text{Coupling Analysis} &: O(N^2) \\
\text{S-Entropy Transform} &: O(N)
\end{align}
where $N$ represents the number of data points.
\end{definition}

This comprehensive methodology framework enabled systematic analysis of consumer-grade sensor data through the multi-scale oscillatory framework, providing robust quantification of activity patterns and their coupling relationships across temporal and physiological scales.
