\section{Theoretical Framework}

\subsection{Consumer-Grade Sensor Data Characteristics}

Consumer-grade wearable devices provide continuous physiological monitoring through photoplethysmography (PPG) and tri-axial accelerometry sensors. PPG sensors measure volumetric changes in blood flow through optical absorption at wavelengths typically between 525-590 nm (green light) or 660-940 nm (red/infrared light). The sampling frequency $f_s$ ranges from 25-100 Hz for PPG signals and 50-200 Hz for accelerometry data.

\begin{definition}[PPG Signal Representation]
The raw PPG signal is represented as:
\begin{equation}
\text{PPG}(t) = I_0 \exp(-\epsilon(\lambda) \cdot c \cdot d(t))
\end{equation}
where $I_0$ denotes incident light intensity, $\epsilon(\lambda)$ represents wavelength-specific absorption coefficient, $c$ indicates blood chromophore concentration, and $d(t)$ denotes time-varying optical path length through tissue.
\end{definition}

\begin{definition}[Accelerometry Signal Representation]
Tri-axial accelerometry data is represented as:
\begin{equation}
\mathbf{a}(t) = [a_x(t), a_y(t), a_z(t)]^T
\end{equation}
where $a_x(t)$, $a_y(t)$, and $a_z(t)$ represent acceleration components along orthogonal axes measured in units of gravitational acceleration $g = 9.81 \text{ m/s}^2$.
\end{definition}

\subsection{Measurement Periodicity and Oscillatory Characteristics}

Consumer-grade sensors exhibit inherent measurement periodicity arising from physiological oscillations, device sampling characteristics, and environmental coupling. The fundamental measurement period $T_{\text{base}}$ is determined by the sensor sampling interval:

\begin{equation}
T_{\text{base}} = \frac{1}{f_s}
\end{equation}

However, physiological signals contain multiple oscillatory components spanning several orders of magnitude in frequency:

\begin{definition}[Physiological Oscillatory Spectrum]
The complete physiological oscillatory spectrum captured by consumer sensors encompasses:
\begin{align}
f_{\text{cardiac}} &\sim 0.8-3.0 \text{ Hz} \quad \text{(heart rate)} \\
f_{\text{respiratory}} &\sim 0.1-0.5 \text{ Hz} \quad \text{(breathing rate)} \\
f_{\text{movement}} &\sim 0.5-15 \text{ Hz} \quad \text{(locomotor activity)} \\
f_{\text{circadian}} &\sim 1.16 \times 10^{-5} \text{ Hz} \quad \text{(24-hour cycles)} \\
f_{\text{ultradian}} &\sim 1.39-2.31 \times 10^{-4} \text{ Hz} \quad \text{(90-120 minute cycles)}
\end{align}
\end{definition}

\subsection{Activity Quantification Through Oscillatory Analysis}

Traditional activity metrics such as step counts and energy expenditure represent time-averaged quantities that obscure underlying oscillatory dynamics. We define activity through its oscillatory characteristics:

\begin{definition}[Oscillatory Activity Representation]
Activity at time $t$ is represented as a multi-component oscillatory signal:
\begin{equation}
A(t) = A_0 + \sum_{k=1}^{N} A_k \cos(2\pi f_k t + \phi_k) + \eta(t)
\end{equation}
where $A_0$ denotes baseline activity level, $A_k$ represents amplitude of oscillatory component $k$, $f_k$ indicates frequency of component $k$, $\phi_k$ denotes phase of component $k$, and $\eta(t)$ represents measurement noise.
\end{definition}

\subsection{Metabolic Equivalent Task (MET) Oscillatory Dynamics}

Metabolic Equivalent of Task (MET) values quantify energy expenditure relative to resting metabolic rate, where 1 MET = 3.5 mL O₂/kg/min. Consumer devices estimate MET values through accelerometry-based algorithms:

\begin{definition}[MET Estimation from Accelerometry]
MET values are estimated through:
\begin{equation}
\text{MET}(t) = \text{MET}_{\text{rest}} + \alpha \cdot \|\mathbf{a}(t)\| + \beta \cdot \text{VAR}(\mathbf{a}(t))
\end{equation}
where $\text{MET}_{\text{rest}} = 1.0$ represents resting metabolic rate, $\alpha$ and $\beta$ denote calibration coefficients, $\|\mathbf{a}(t)\|$ indicates vector magnitude of acceleration, and $\text{VAR}(\mathbf{a}(t))$ represents acceleration variance over a sliding time window.
\end{definition}

\subsection{Heart Rate Variability as Oscillatory Coupling Indicator}

Heart rate variability (HRV) metrics derived from PPG signals provide direct measurements of autonomic nervous system oscillatory coupling:

\begin{definition}[RMSSD Calculation]
Root Mean Square of Successive Differences (RMSSD) quantifies short-term HRV:
\begin{equation}
\text{RMSSD} = \sqrt{\frac{1}{N-1} \sum_{i=1}^{N-1} (\text{RR}_{i+1} - \text{RR}_i)^2}
\end{equation}
where $\text{RR}_i$ represents the $i$-th inter-beat interval measured in milliseconds, and $N$ denotes the total number of intervals.
\end{definition}

\begin{definition}[SDNN Calculation]
Standard Deviation of NN intervals (SDNN) quantifies overall HRV:
\begin{equation}
\text{SDNN} = \sqrt{\frac{1}{N-1} \sum_{i=1}^{N} (\text{RR}_i - \overline{\text{RR}})^2}
\end{equation}
where $\overline{\text{RR}}$ represents the mean inter-beat interval.
\end{definition}

\subsection{Temporal Resolution and Measurement Windows}

Consumer devices typically provide aggregated measurements over fixed temporal windows. Standard temporal resolutions include:

\begin{itemize}
\item 1-minute windows: Heart rate, step counts, movement intensity
\item 5-minute windows: HRV metrics, detailed activity classification
\item 15-minute windows: Sleep stage estimation, circadian rhythm analysis
\item 1-hour windows: Energy expenditure, long-term trend analysis
\end{itemize}

\begin{definition}[Temporal Window Function]
For a measurement window of duration $\Delta t$, the windowed signal is:
\begin{equation}
X_{\text{window}}[n] = \frac{1}{\Delta t} \int_{n\Delta t}^{(n+1)\Delta t} X(t) \, dt
\end{equation}
where $X(t)$ represents the continuous physiological signal and $X_{\text{window}}[n]$ denotes the $n$-th windowed measurement.
\end{definition}

\subsection{Data Imprecision and Oscillatory Robustness}

Consumer-grade sensors exhibit measurement imprecision due to motion artifacts, sensor placement variability, and environmental factors. However, oscillatory analysis provides robustness to measurement noise through frequency-domain filtering and phase relationship preservation:

\begin{theorem}[Oscillatory Measurement Robustness]
For a noisy measurement $\tilde{X}(t) = X(t) + \epsilon(t)$ where $\epsilon(t)$ represents additive noise with power spectral density $S_\epsilon(f)$, the oscillatory components remain recoverable provided:
\begin{equation}
\frac{S_X(f_k)}{S_\epsilon(f_k)} > \gamma_{\text{threshold}}
\end{equation}
where $S_X(f_k)$ denotes signal power at frequency $f_k$ and $\gamma_{\text{threshold}}$ represents the minimum signal-to-noise ratio for reliable oscillatory component extraction.
\end{theorem}

This robustness enables extraction of meaningful oscillatory patterns from imprecise consumer-grade measurements, forming the foundation for multi-scale oscillatory analysis of human activity patterns.
